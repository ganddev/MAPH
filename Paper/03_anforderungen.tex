   	\section{Anforderungen}
   	Im Folgenden Abschnitt werden die Anforderungen für die LIAR Android App aufgelistet.
   	
   	\subsection{Muss-Kriterien}
	\begin{itemize}
	\item{}Die App soll mit dem mobilen Betriebssystem Android entwickelt werden.
	\item{}Ein EEG- und eine Galvanic Skin Sensor (Hautleitwert) sollen Daten an das Android Smartphone senden können.
	\item{}Die Messwerte des EEG-Sensors und der Hautleitwert des Benutzers sollen auf dem Smartphone ausgelesen werden können.
	\item{}Die Messwerte des Benutzers können ausgewertet und angezeigt werden.
	\item{}Es kann ein Benutzerprofil erstellt werden.
	\item{}Es lässt sich eine neue Spielsession (Spielsession, Spieldauer, Spieleranzahl) erstellen.
	\item{}Das Speichern der Messwertauswertung je Benutzer ist möglich.
	\item{}Eine Spielsession kann durchgeführt werden.
	\item{}Es kann ein Lügendetektortest mit vorgegebenen Fragen absolviert werden.
 werden.
	\item{}Die Spielergebnisse werden angezeigt und können gespeichert werden.
	\end{itemize}		
	
	\subsection{Kann-Kriterien}
	\begin{itemize}
		\item{}Die Anzeige der Messwerte des EEG-Sensors und der Hautleitwert des Benutzers soll grafisch in einem dynamischen XY-Diagramm erfolgen.
		\item{}Es kann ein Lügendetektortest mit eigenen Fragen absolviert werden.
		\item{}Die Spielergebnisse können auf sozialen Netzwerken veröffentlicht werden.	
	\end{itemize}

